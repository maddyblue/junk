\documentclass{article}
\usepackage[top=1in, bottom=1in, left=1in, right=1in]{geometry}
\usepackage{amsmath}
\usepackage{graphicx}
\begin{document}

\begin{flushright}
Matt Jibson \\
ST 521 \\
HW 5
\end{flushright}

\begin{enumerate}
	\item %1
		\begin{enumerate}
			\item $S = \{0, 1, 2, 3, 4\}$
				\begin{displaymath}
					P = \left( \begin{array}{ccccc}
						0 & 1/3 & 2/3 & 0 & 0 \\
						1/2 & 0 & 1/6 & 1/3 & 0 \\
						0 & 1/2 & 0 & 1/6 & 1/3 \\
						0 & 0 & 3/4 & 0 & 1/4 \\
						0 & 0 & 0 & 1 & 0
					\end{array} \right)
				\end{displaymath}
			\item Write down our initial equations:
				\begin{displaymath}
					\begin{array}{l}
						\Pi_0 = \frac{1}{2} \Pi_1 \\
						\Pi_1 = \frac{1}{3} \Pi_0 + \frac{1}{2} \Pi_2 \\
						\Pi_2 = \frac{2}{3} \Pi_0 + \frac{1}{6} \Pi_1 + \frac{3}{4} \Pi_3 \\
						\Pi_3 = \frac{1}{3} \Pi_1 + \frac{1}{6} \Pi_2 + \Pi_4 \\
						\Pi_4 = \frac{1}{3} \Pi_2 + \frac{1}{4} \Pi_3
					\end{array}
				\end{displaymath}
				Write everything in terms of $\Pi_0$:
				\begin{displaymath}
					\begin{array}{l}
						\Pi_0 = \Pi_0 \\
						\Pi_1 = 2 \Pi_0 \\
						\Pi_2 = \frac{10}{3} \Pi_0 \\
						\Pi_3 = \frac{28}{9} \Pi_0 \\
						\Pi_4 = \frac{17}{9} \Pi_0
					\end{array}
				\end{displaymath}
				Double checking with $\Pi_4 = \frac{1}{3} \Pi_2 + \frac{1}{4} \Pi_3$ yields $\frac{17}{9}$ on both sides, so things look correct so far. Now sum everything and solve: $\Pi_0 + 2 \Pi_0 + \frac{10}{3} \Pi_0 + \frac{28}{9} \Pi_0 + \frac{17}{9} \Pi_0 = 1 \Rightarrow \frac{34}{3} \Pi_0 = 1 \Rightarrow \Pi_0 = \frac{3}{34}$. So, $\Pi = (3/34, 3/17, 5/17, 14/51, 1/6)$.
			\item There is a $3/34$ chance she will find it empty and a $1/6$ chance she will find it full.
		\end{enumerate}
	\item %2
		\begin{enumerate}
			\item For states 0, 1, 2, 3 the smallest number of steps is 4. For states 4, 5 the smallest number of steps is 6.
			\item
			\item
		\end{enumerate}
\end{enumerate}

\end{document}