\chapter{Discussion and Conclusion}

The results above show that the chip performs well under the anticipated conditions for integrated biosensing (i.e., low concentrations of solution). The design hypotheses of the chip were tested, and some were found to be conclusive. Specifically, it is clear that 1) output increases with increased WE area, and 2) AE area should be much larger than WE area to prevent saturation. There is not conclusive evidence to support the hypotheses that distance from WE to AE, or the shape of the sensor site has any major impact on the result. Thus, further chips designed should have large WEs, and larger AEs.

We fabricated a chip with an array of electrochemical sensor cells of varying make. This chip was characterized to find high-performing electrodes which were then each characterized by the response to various concentrations of expected chemicals that would be released by living cells. In addition, more characterizations were done to test the hypotheses put forward during initial chip design, and some conclusions were drawn from the results which led to design goals of future chips.
