\chapter{Introduction}

Neurotransmitters play an important role in central nervous systems. Of interest is detecting molecular gradients that are essential in the development of tissue and organ systems. Nitric oxide, a neurotransmitter, is important in this development. Molecular gradients are difficult to detect because of the relative large size of the cells compared to the electrochemical sensors used in sensing systems. Furthermore, in order to detect a gradient, an sensor array must be used in order to collect real-time spatial data. Previous work described the methodology of creating such a sensor array based on CMOS technology. This chip has been fabricated. This paper describes the characterization of the chip in response to expected signals from living cells and presents a potentiostat suitable for this work to be used in a future chip.
