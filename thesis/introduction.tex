\chapter{Introduction}

Neurotransmitters play an important role in central nervous systems. Of interest is detecting molecular gradients that are essential in the development of tissue and organ systems. Nitric oxide, a neurotransmitter, is important in this development. Molecular gradients are difficult to detect because of the relative large size of the cells compared to the electrochemical sensors used in sensing systems. Furthermore, in order to detect a gradient, an sensor array must be used in order to collect real-time spatial data. Due to this requirement of a sensor array, it is difficult to construct a device with discrete parts, since it would be quite large. Thus, an integrated sensor must be constructed. Integration allows components to be small enough to have many sensors in the area of a cell, and is thus able to sense a chemical image, or gradient.

The smaller a sensor is, the greater temporal and spatial resolution is possible since more sensors can be put into an identical area. This miniaturization allows finer gradients to be measured, and thus better conclusions can be drawn. However, miniaturization requires further smaller components and fabrication processes. This goal of a fine gradient continues to push interest and research (like this work) toward integrated sensors.

This thesis is one step toward that goal of integrated sensors, specifically an integrated biosensor. Previous work in this lab has resulted in the design and subsequent fabrication of a chip with integrated electrodes \cite{karegar2007ema}. The design of this chip had a number of hypotheses about electrode construction, and used many designs to test them. This thesis discusses those tests, their results, and conclusions drawn. Hence, this thesis is one step toward the goal of integrated biosensing.
