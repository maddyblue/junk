\chapter{Introduction}

\section{Neurotransmitter Detection}

Neurotransmitters play an important role in central nervous systems. Nitric oxide (NO), a neurotransmitter, is important in this development \cite{bicker2005sag, bulotta2005css, contestabile2004rno}. Of interest is detecting molecular gradients that are essential in the development of tissue and organ systems \cite{wolpert1996ohy, gurdon2001mgi}. Molecular gradients are difficult to detect because of the relative large size of the cells compared to the electrochemical sensors used in sensing systems. Furthermore, in order to detect a gradient, an sensor array must be used in order to collect real-time spatial data. Due to this requirement of a sensor array, it is difficult to construct a device with discrete parts, since it would be quite large. Thus, an integrated sensor must be constructed. Integration allows components to be small enough to have many sensors in the area of a cell, and is thus able to sense a chemical image, or gradient.

For example, in order to construct a discrete sensor apparatus, one would first have to acquire individual electrodes on the micron scale, and a device able to hold them in place a few microns apart. This is theoretically possible using a probestation and micromanipulators. Next, some sort of covering would have to be placed sufficient that the cell slice could be mounted onto it, the sensors would be in contact with the slice, and the nutrient solution around the slice would be contained. This is theoretically possible by constructing a well and lowering the many pins (discussed next) into the well. In order to measure a gradient, let us assume a $2 \times 2$ array of sensors is used, which is 4 sensors, at 3 electrodes each, or 12 electrodes. (For comparison, our chip has more than 84 sensors, around 200 electrodes.) Positioning twelve micromanipulators on one probestation is unlikely to happen (six is difficult to do). Furthermore, lowering that many pins into a well without accidentally shorting them, while getting them all within a few microns of eachother, will not happen reliably (if ever). Assuming all of this did happen, the gradient measured is too small to be useful due to the small array size. In addition, one must have a potentiostat for every sensor. Potentiostats can cost a few thousand dollars and take up about half the volume of a normal computer case. Stacking 84 potentiostats near eachother is prohibitively expensive and logistically unmanageable. Thus, discrete sensors are not practical for this work.

Integrated sensors do not have any of the described problems that discrete sensors do. Sensors can be mounted on chip in a location able to support an ovary and its surrounding solutions. Potentiostats can be integrated in the chip for each sensor \cite{murari2005ipn, stanacevic2007vpa, zhang2005eam}. The smaller a sensor is, the greater temporal and spatial resolution is possible since more sensors can be put into an identical area. This miniaturization allows finer gradients to be measured, and thus better conclusions can be drawn. However, miniaturization requires further smaller components and fabrication processes. This goal of a fine gradient continues to push interest and research (like this work) toward integrated sensors.

\section{Objectives}

This thesis is one step toward that goal of integrated sensors, specifically an integrated biosensor. Previous work in this lab has resulted in the design and subsequent fabrication of a chip with integrated electrodes \cite{karegar2007ema}. The design of this chip had a number of hypotheses about electrode construction, and used many designs to test them. This thesis discusses those tests, their results, and conclusions. Specifically, we test the effects of varying shape, size, distance and configuration of electrodes, and are able to draw conclusions about the importance of each of those aspects.

The next chapter discusses the scientific background to the research as well as the previous and current work in the area of integrated biosensors. Chapter 3 documents the design and design hypotheses of the chip, the tests performed, and their results, drawing conclusions about the chip design hypotheses. Chapter 4 gives some proof-of-concept tests and results while testing live ovary slices. Chapter 5 draws conclusions about future work.
