\chapter{Experimental Setup}

All experiments were done on a Micromanipulator, Inc.\ Probestation. Probe tips and other objects (e.g., pipette tips, thermometer) were held in place by micromanipulators. All electrochemistry experiments were done with a CH Instruments 1207 potentiostat.

Fabrication was done at Avago, Inc., which produced two 6-inch wafers. The wafers were diced by Aspen Technologies to separate the chips. To prepare a single chip, a well of PDMS was constructed such that the exterior would not exceed the edge of the chip and would allow for pins to be lowered onto the probe pads, and the interior would allow the sensor array to be exposed (Fig. \ref{chip-diagram}). This was done so that solution and ovaries could be put over the sensor array with no interference to the probe pads. The well was attached to the chip with glue, producing a seal that would not allow leakage of the interior solution to the probe pads. The well was high enough so that the interior could hold at least 70mL of solution. The interior was wide enough so that ovaries could be easily attached to the surface.

\begin{figure} %{{{ chip-diagram
\centering
\setlength{\unitlength}{0.007 \linewidth}
\begin{picture}(100, 100)
	% 1x1 square
	\newsavebox{\sensor}
	\savebox{\sensor}(1, 1){
		\multiput(0, 0)(0, 1){2}{\line(1, 0){1}}
		\multiput(0, 0)(1, 0){2}{\line(0, 1){1}}
	}

	% 5x5 sensor grid
	\multiput(45, 45.5)(0, 2){5}{\usebox{\sensor}}
	\multiput(47, 45.5)(0, 2){5}{\usebox{\sensor}}
	\multiput(49, 45.5)(0, 2){5}{\usebox{\sensor}}
	\multiput(51, 45.5)(0, 2){5}{\usebox{\sensor}}
	\multiput(53, 45.5)(0, 2){5}{\usebox{\sensor}}

	% vertical pads
	\multiput(0, 0.5)(0, 2){50}{\usebox{\sensor}}
	\multiput(98, 0.5)(0, 2){50}{\usebox{\sensor}}

	% horizontal pads
	\multiput(2, 0.5)(2, 0){48}{\usebox{\sensor}}
	\multiput(2, 98.5)(2, 0){48}{\usebox{\sensor}}

	% exterior
	\multiput(0, 0)(0, 100){2}{\line(100, 0){100}}
	\multiput(0, 0)(100, 0){2}{\line(0, 100){100}}

	% well exterior
	\multiput(2, 2)(0, 96){2}{\line(96, 0){96}}
	\multiput(2, 2)(96, 0){2}{\line(0, 96){96}}

	% well interior
	\put(50, 50){\circle{15}}

	\put(10, 10){probe pads}
	\put(10, 10){\vector(-1, -1){8.5}}

	\put(35, 35){$5 \times 5$ sensor array}
	\put(50, 39){\vector(0, 1){11}}

	\put(35, 65){well interior edge}
	\put(50, 64){\vector(0, -1){6.5}}

	\put(10, 89){well exterior edge}
	\put(25, 93){\vector(0, 1){5}}

	\put(70, 89){chip edge}
	\put(80, 93){\vector(0, 1){7}}
\end{picture}
\caption{Diagram of chip with PDMS well from above.}
\label{chip-diagram}
\end{figure} %}}}

The first characterization of the chip was needed to find the best electrode cell configuration, i.e., the cell with the largest response for a given input. Recall that there are 25 sensor areas. Each area contains multiple working electrodes, and one or more reference and auxiliary electrodes. Two or more working electrodes were tested at each sensor area. Multiple CV runs were performed on each electrode from 0.5V to -0.5V at 0.1V/s in 1M NaCl. The value of an individual run was taken to be the magnitude of the difference of the averages of the two potential sweep directions between -0.2V and 0.2V. That is, the average value from 0.2V to -0.2V in one potential sweep direction was found; the average value from 0.2V to -0.2V in the other potential sweep direction was found; the difference of these averages was found by subtraction. Finally, all results were grouped and averaged per sensor and the sensor area corresponding to the largest result was taken to be the best electrode cell configuration.

To find the ideal potential at which to run the amperometry characterization, a hydrodinamic voltammogram (HDV) was performed to find the optimum potential for amperometry. This was done by running multiple amperometry experiments at identical conditions except for potential, which was varied between 0.6V and 1.25V. In order to ensure identical conditions, the electrode was cleaned before each run by CV from 1.5V to -0.8V at 1V/s for 100 cycles in $\mathrm{H}_2 \mathrm{SO}_4$. The chip well was filled with 50$\mu$L 0.1M KCl. A syringe of 600$\mu$M norepinefrin (NE) was loaded into a pump. The end of the syringe was connected to a pipette, the end of which was lowered into the well so that it was below the surface of the solution. An ameprometry experiment was carried out where the syringe pump would be enabled for 10 seconds at 100$\mu$L/min. The value of the experiment was taken to be the difference between the idle point just before the syringe pump was enabled (generally close to 0) and the lowest point of the resulting curve. The optimum potential was found by taking the largest result.

The final stage of characterization was done to build a find the current generated by a given concentration. This was done using similar conditions as in the HDV characterization. The electrode was cleaned using the CV process with $\mathrm{H}_2 \mathrm{SO}_4$ as listed above. The well was filled with 50$\mu$L 0.1M KCl. A syringe was filled with various concentrations of NE, to which a pipette was connected to the end and lowered into the solution belowe the surface. Amperometry was performed at 0.85V. The syringe pump was enabled for 10 seconds at 100$\mu$L/min after the potentiostat read a stable value close to zero. The value was taken to be the difference between the idle zero point and the lowest point of the resulting curve. This was repeated for each concentration.

Furthermore, the previous experiment was done using both 2 and 4 working electrodes shorted together in parallel (which was possible due to the common reference and auxiliary electrodes for the sensor area we were using). This created a larger effective surface area on the working electrode, theoretically able to sustain a higher current. To verify that this was a valid technique CVs were performed using 100$\mu$M NE in neurobasal (NB) from -0.2V to 1.0V at 0.025V/s. The value was taken to be the lowest recorded point.

Many ovary slices were tested throughout characterization experiments. These were done by preparing a chip with a well and testing it for basic functionality by performing an arbitrary CV. Ovaries were extracted from mice and attached to the surface of the chip over the sensor array, covered in vitrogen. Neurebasal was periodically added to keep the cell from dying and drying out. The ovaries and chips were kept at $37\,^{\circ}\mathrm{C}$ in order to keep them alive. A heat lamp was used during testing with a thermometer measuring temperature directly above the well in an effort to keep the temperature at $37\,^{\circ}\mathrm{C}$. Amperometry tests were run at $0.85V$ for 30 minutes. After the main test, three injections were performed, each of 20$\mu$L: (1) a blank of NB to test for injection noise, (2) 25$\mu$M NE in NB to test the chip's current response to NE, (3) a 1:1 ratio by volume of KCl:NB to release all remaining NE from the cell, which would provide the maximum expected signal from the main 30 minute block.