\begin{abstract}

Neurotransmitters play an important role in central nervous systems. Nitric oxide, a neurotransmitter, is important in this development. Of interest is detecting molecular gradients that are essential in the development of tissue and organ systems. Molecular gradients are difficult to detect because of the relative large size of the cells compared to the electrochemical sensors used in sensing systems. Furthermore, in order to detect a gradient, an sensor array must be used in order to collect real-time spatial data. Due to this requirement of a sensor array, it is difficult to construct a device with discrete parts, since it would be quite large. Thus, an integrated sensor must be constructed. Integration allows components to be small enough to have many sensors in the area of a cell, and is thus able to sense a chemical image, or gradient.

Previous work has resulted in the production of a chip with an array of 21 sensor sites of individual and specific design with the purpose of testing hypotheses relating the shape, size, distance and configuration to the output signal strength. The electrodes are on the micron scale, and are capable of performing electrochemistry on living cells. The sensor sites were characterized using differential pulse voltammetry to find their relative performance. Based on these results, further tests were performed to test hypotheses regarding the shape, size, distance and configuration of the electrodes. The lower detection limit is found on two of the best sensors. A proof-of-concept test is done with a living mouse-ovary slice, which showed results similar to those in the literature.

Results show that the important design characteristics are working-electrode size (larger is better), and the ratio of the areas of the working to auxiliary electrode (smaller ratio is better). The other design characteristics (distance, shape, configuration) played, in general, did not have much impact on the output. Conclusions about the design of future chips is made based on these findings.

\end{abstract}
