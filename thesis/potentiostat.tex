\chapter{Potentiostat Design}

The chip characterized thus far is an array with 21 sensor sites, each with 4 electrode sets (that is, auxilliary, reference, and working electrodes). The future goal of this work is to have a similar electrode configuration over the entire array, which will be somewhere above 80 electrode sets. This could require a large external probe station on which to make these measurements, with each electrode set having its own expensive potentiostat. This is obviously either prohibitively expensive, logistically impossible, or simply foolish. Instead, the future chip should have individual potentiostats for every electrode set. Since we are attempting to measure signals in the picoamp and microsecond range, these potentiostats must be low noise (on the order of femtoamps) and have bandwidth of at least 1MHz \cite{mosharok2005aee}.

A basic potentiostat consists of two parts: a voltage follower and a current-to-voltage converter \cite{kissinger1996iai}. This design focuses on the current-to-voltage converter, which must take picoamp input signals and output a voltage measureable by an analog-to-digital converter (ADC) that can change every microsecond. Previous work has used oversampling with delta-sigma modulators \cite{murari2005ipn} \cite{stanacevic2007vpa}. Oversampling requires sampling the same input many times (up to a few seconds in these cases) in order to read lower concentrations with low (or no) amplification to the input signal. The lack of amplification reduces noise, but greatly decreases the bandwidth of the detector due to the oversampling time. Thus, a solution not based on oversampling is needed. Our solution was to use ultra-low input current operational ampfiliers, with a bandwidth of at least 1MHz, to amplify the signal to the order of hundreds of millivolts with a time resolution similar to the input.
