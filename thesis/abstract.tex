\begin{abstract}

Previous work has resulted in the production of a chip with an array of 21 sensor sites of individual design, with electrodes on the micron scale, which are capable of performing electrochemistry on living cells. The sensor sites were characterized using differential pulse voltammetry to find their relative performance. Based on these results, further tests were performed to test hypotheses regarding the size, shape, and spacing of electrodes. Results show some of these to be sound, and others to be inconclusive or incorrect. The lower detection limit is found on two of the best sensors. A proof-of-concept test is done with a living mouse-ovary slice, which showed results similar to those in the literature. Conclusions about the design of future chips is made based on these findings.

\end{abstract}
