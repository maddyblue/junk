\begin{abstract}

Advancing our understanding of how the central nervous system works under specific conditions requires real-time, simultaneous detection of a number of key signaling molecules, like nitric oxide (NO). NO diffuses widely and rapidly, has a lifetime in milliseconds, and presents among other high concentration, interfering compounds in the nanomolar range in most biological systems. Current microelectrode-based electrochemical NO sensors have diameters in the micrometer range, much larger than biological cells which are in the micron range, and are thus insufficient for analyzing cell-to-cell interactions. A chip was fabricated using the 0.6$\mu$m CMOS process to overcome these difficulties, with a $5 \times 5$ array of sensors in the 2$\mu$m range. Characterization results of the chip with norepinephrin indicate sensitivity into the 20$\mu$M range and suggest higher-sensitivity devices can offer improvement.

\end{abstract}