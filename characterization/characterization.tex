\documentclass{article}
\usepackage[top=1in, bottom=1in, left=1in, right=1in]{geometry}
\usepackage{graphicx}
\begin{document}


\title{Electrochemical Biosensor Array Characterization}
\author{Matt Jibson}

\maketitle{}

\begin{abstract}

Advancing our understanding of how the central nervous system works under specific conditions requires real-time, simultaneous detection of a number of key signaling molecules, like nitric oxide (NO). NO diffuses widely and rapidly, has a lifetime in milliseconds, and presents among other high concertration, interfering compounds in the nanomolar range in most biological systems. Current microelectrode-based electrochemical NO sensors have diameters in the micrometer range, much larger than biological cells which are in the micron range, and are thus insufficient for analyzing cell-to-cell interactions. A chip was fabricated using the $0.6\mu m$ CMOS process to overcome these difficulties, with a $5 \times 5$ array of sensors in the 2$\mu m$ range. Characterazition results of the chip indicate sensitivity into the 20$\mu$ molar range and suggest higher-sensitivity devices can offer improvement.

\end{abstract}

\section{Introduction}

\section{Background}

\section{Experimental Setup}

\section{Experimental Results}

\section{Conclusion}

%1: intro - about biosensing
%2: background - about chemicals, how they interact at sensor surface, how electron gets exchanged, how current gets generated, sensor surface preparation: involves adding nafion, cleaning, etc.
%3: basic bg on silicon technology, details about our sensor chip, can be linked to section 2
%4: experimental setup: chips, diced, wells (rings), micro pumps, potentiostat, amperometry. limitations of this setup, what are we measuring with this setup, probe setup has resistance, capacitance, inductance. look at literature, try to find characterization of that. probe tip on RLC filter? maybe sensor can detect nanomolar fine. figure out attenuation out of sensor probes.
%5: experimental results. HDV, 2-pin, 4-pin, (1-pin?), ovaries
%6: conclusions.

\end{document}
