\documentclass{article}
\usepackage[top=1in, bottom=1in, left=1in, right=1in]{geometry}
\usepackage{graphicx}
\usepackage{amsmath}
\begin{document}

\begin{flushright}
Matt Jibson \\
EG 520 \\
HW 6
\end{flushright}

\begin{itemize}
	\item[17.3]
		\begin{enumerate}
			\item[a.] Problem:
				\begin{displaymath}
					\begin{array}{rl}
						\textrm{maximize} & 2x_1 + 3x_2 \\
						\textrm{subject to} & x_1 + 2x_2 \le 4 \\
						& 2x_1 + x_2 \le 5 \\
						& x_1, x_2 \ge 0
					\end{array}
				\end{displaymath}
				Solution: $x_1 = 2, x_2 = 1$ with objective value = 7.
			\item[b.] Dual is:
				\begin{displaymath}
					\begin{array}{rl}
						\textrm{minimize} & 4y_1 + 5x_2 \\
						\textrm{subject to} & y_1 + 2y_2 \le 2 \\
						& 2y_1 + y_2 \le 3 \\
						& y_1, y_2 \ge 0
					\end{array}
				\end{displaymath}
				Solution: $y_1 = 1 \frac{1}{3}, x_2 = \frac{1}{3}$ with objective value = 7.
		\end{enumerate}
	\item[17.6]
		\begin{enumerate}
			\item[a.] Dual is:
				\begin{displaymath}
					\begin{array}{rl}
						\textrm{maximize} & \lambda_1 + \cdots + \lambda_n \\
						\textrm{subject to} & a_1 \lambda_1 \le 1 \\
						& \vdots \\
						& a_n \lambda_n \le 1 \\
						\textrm{where} & 0 < a_1 < \cdots < a_n
					\end{array}
				\end{displaymath}
			\item[b.] Duality theorem: If the primal problem has an optimal solution, then so does the dual, and the optimal values of their respective objective functions are equal.
		\end{enumerate}
	\item[17.9] If $\boldsymbol{\mu}^T \boldsymbol{x} = 0$ then atleast one element of $\boldsymbol{\mu}$ or $\boldsymbol{x}$ per pair is 0.
	\item[19.6a] Problem:
		\begin{displaymath}
			\begin{array}{rl}
				\textrm{minimize} & 2x_1 + 3x_2 - 4 \\
				\textrm{subject to} & x_1 x_2 = 6
			\end{array}
		\end{displaymath}
		Thus $\nabla f(\boldsymbol{x}) = [2, 3]^T, \nabla h(\boldsymbol{x}) = [x_2, x_1]^T$. So the minimizers satisfy:
		\begin{eqnarray}
			2 + \lambda x_2 & = & 0 \nonumber \\
			3 + \lambda x_1 & = & 0 \nonumber \\
			x_1 x_2 & = & 6 \nonumber
		\end{eqnarray}
		Clearly $\lambda, x_1, x_2$ must all be nonzero. If any are then the equations would be $2 = 0$ or $3 = 0$. Solving the first equation for $x_2$ yields $x_2 = \frac{-2}{\lambda}$. Solving the second equation for $x_1$ yields $x_1 = \frac{-3}{\lambda}$. Substituting these into the third equation yields $\frac{-2}{\lambda} \frac{-3}{\lambda} = 6 \Rightarrow \frac{6}{\lambda^2} = 6 \Rightarrow \lambda^2 = 1 \Rightarrow \lambda = 1, -1$. For $\lambda = 1: x_2 = -2, x_1 = -3$. For $\lambda = -1: x_2 = 2, x_1 = 3$. Analysis is the same if this problem is changed to a maximization problem.
	\item[19.10] Problem:
		\begin{displaymath}
			\begin{array}{rl}
				\textrm{maximize} & ax_1 + bx_2 \\
				\textrm{subject to} & x_1^2 + x_2^2 = 2
			\end{array}
		\end{displaymath}
		Thus $\nabla f(\boldsymbol{x}) = [a, b]^T, \nabla h(\boldsymbol{x}) = [2x_1, 2x_2]^T$. So the maximizers satisfy:
		\begin{eqnarray}
			a + 2 \lambda x_1 & = & 0 \nonumber \\
			b + 2 \lambda x_2 & = & 0 \nonumber \\
			x_1^2 + x_2^2 & = & 2 \nonumber
		\end{eqnarray}
		Given the solution of $[1, 1]^T$, $a + 2 \lambda (1) = 0, b + 2 \lambda (1) = 0 \Rightarrow a + 2 \lambda = 0, b + 2 \lambda = 0 \Rightarrow a + 2 \lambda = b + 2 \lambda \Rightarrow a = b$.
	\item[19.11a] Problem:
		\begin{displaymath}
			\begin{array}{rl}
				\textrm{maximize} & x_1 x_2 - 2x_1 \\
				\textrm{subject to} & x_1^2 - x_2^2 = 0
			\end{array}
		\end{displaymath}
		Thus $\nabla f(\boldsymbol{x}) = [x_2 - 2, x_1]^T, \nabla h(\boldsymbol{x}) = [2x_1, -2x_2]^T$. So the minimizers satisfy:
		\begin{eqnarray}
			x_2 - 2 + 2 \lambda x_1 & = & 0 \nonumber \\
			x_1 - 2 \lambda x_2 & = & 0 \nonumber \\
			x_1^2 - x_2^2 & = & 0 \nonumber
		\end{eqnarray}
		From the third equation, $x_1^2 = x_2^2$, which is satisfied by both $[1, 1]^T$ and $[-1, 1]^T$.
		
		Solving the first equation for $x_2$ yields $x_2 = 2 - 2 \lambda x_1$. Solving the second equation for $x_1$ yields $x_1 = 2 \lambda x_2$. Substituting yields $x_2 = 2 - 2 \lambda (2 \lambda x_2) = 2 - 4 \lambda^2 x_2 \Rightarrow 2 = (1 + 4 \lambda^2) x_2 \Rightarrow x_2 = \frac{2}{1 + 4 \lambda^2} \Rightarrow 2 - 2 \lambda x_1 = \frac{2}{1 + 4 \lambda^2}$.
	\item[19.15a]
\end{itemize}

\end{document}