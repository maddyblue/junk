\documentclass{article}
\usepackage[top=1in, bottom=1in, left=1in, right=1in]{geometry}
\begin{document}

\begin{flushright}
Matt Jibson \\
MU 437 \\
Final
\end{flushright}

{\setlength{\baselineskip}{1.6\baselineskip}

The first organ is French. This is clear from the stop and division names and abundance of reeds, harmonic flutes and celestes, and relative little upper work. Each division is very similar to the others. For example, all divisions have an 8' trompette and 4' clairon. Each manual division has a harmonic flute, 8' bourdon, 8' viole de gambe, and 2' principal. Each division has one or two unique reeds (bombarde, clarinette, basson-hautbois, basson), but these are not enough to significantly differentiate the divisions. There is an attempt to conform to some sort of ``werk principle'' guide, with 8', 4', and 2' principals in the manuals, but these are not accompanied by much upper work, and are not always the lowest pitched principal of the division (e.g., the 8' montre in the positif below the 4' prestant). Hence, this organ is most suited for monophonic (i.e., not counterpoint), work from the French Romantic period.

The second organ is American. This is apparent from the mix of styles in the different divisions, as well as the names of the divisions and stops. In the great and positiv, a Germon/``werk principle'' style is evident from few reeds (three only in the great), one string, large principal chorus, and plentiful upper work. In the swell and choir, however, we see seven reeds between the two, two celestes and more strings, and only one principal chorus in the swell. The pedal has everything: large principal chorus, many reeds and five pitch levels, and mixtures. This organ appears to have the capability to play all music. There are three manuals and the pedal each with a principal chorus adn good upper work for polyphonic music and counterpoint, and enough reeds, flutes and solo stops to play the French repertoire. A criticism, however, is that this organ does not appear to be very unified, like the organ above. This organ does not have a consistent arrangement of stops between divisions, and appears to be trying to fill in the missing tonal styles from division to division.

The third organ is German. Again, this is evident from the stop and division names and stop arrangement. There is a principal chorus in each division, along with mixtures and upper work. The reeds present in every division are German: fagott, klarine, krummhorn, schalmei, etc. The pedal, specifically, has two 16' reeds: a posaune and fagott, which serve well in Baroque music. This organ has one celeste, serving as the only string. Ergo, this organ is suited best for counterpoint and polyphony, but not for later Romantic music.

The fourth organ is obviously a chamber instrument. Its three flute stops can be used for light accompaniament that probably has little ability to compete with loud choirs or instruments. However, it is ideal for (non-organ) trio sonatas, Baroque and early music, and other situations where a continuo is needed. An appropriate title for this kind of instrument may be ``Ultimate Chamber Accompanimental organ'', or UCA organ, for short.

}

\end{document}
