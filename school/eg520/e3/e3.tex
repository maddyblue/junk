\documentclass{article}
\usepackage[top=1in, bottom=1in, left=1in, right=1in]{geometry}
\usepackage{graphicx}
\usepackage{amsmath}
\begin{document}

\begin{flushright}
Matt Jibson \\
EG 520 \\
Exam 3
\end{flushright}

\begin{enumerate}
	\item
		\begin{enumerate}
			\item This is a twist on the asymmetric form. Dual is:
		 		\begin{displaymath}
					\begin{array}{rl}
						\textrm{minimize} & \boldsymbol{x}^\top \boldsymbol{y} \\
						\textrm{subject to} & \boldsymbol{P} \boldsymbol{y} \le \boldsymbol{e}
					\end{array}
				\end{displaymath}
			\item %Assume that $\boldsymbol{Py} > \boldsymbol{y}$, and take the largest element of $\boldsymbol{y}$ to be $y_i$.
			\item The primal is feasible. Since the dual is not feasible, the primal is unbounded.
			\item Since the primal is feasible and unbounded, there must exist a vector $\boldsymbol{x} \ge 0$ such that $\boldsymbol{x}^\top \boldsymbol{P} = \boldsymbol{x}^\top$ and $\boldsymbol{x}^\top \boldsymbol{e} = 1$.
		\end{enumerate}
	\item Yes, $x^*$ can fail to satisfy the Lagrange condition since no assumption of regularity was made. Hence, we cannot assume that $\nabla h(x^*) \ne 0$. Thus, if $\nabla h(x^*) = 0$, then there is no $\lambda^*$ that satisfies the Lagrange condition since $\nabla f(x^*) \ne 0$.
	\item
		\begin{enumerate}
			\item The point $[1/3, 1/3, 1/3]^\top$ is closest to the plane formed by the three feasible points at $[1/2, 1/2, 1/2]^\top$. Since we are dealing with a convex problem, we can thus conclude that the objective value at $[1/3, 1/3, 1/3]^\top$ is less than 1.
			\item If we know that the three points are global minimizers, then taking the conclusion made in part (a), we know that the point $[1/3, 1/3, 1/3]^\top$ is not feasible since it must have an objective value less than 1, but if the three points are global minimizers then the lowest objective value for the feasible points is 1, and so any point with an objective value less than 1 must be not feasible.
		\end{enumerate}
	\item
		\begin{enumerate}
			\item Since the points are feasible, $g(\boldsymbol{x}) \le 0, \boldsymbol{\mu}_0 \ge 0$. We need to show $f(\boldsymbol{x}_0) \ge f(\boldsymbol{x}_0) + \boldsymbol{\mu}_0 g(\boldsymbol{x})$, or $\boldsymbol{\mu}_0 g(\boldsymbol{x}) \le 0$. Since $g(\boldsymbol{x})$ is positive and $\boldsymbol{\mu}$ negative, this is true.
			\item If $f(\boldsymbol{x}_0) = q(\boldsymbol{\mu}_0)$, then $f(\boldsymbol{x}_0) = f(\boldsymbol{x}_0) + \mu_0 g(\boldsymbol{x})$, so the KKT conditions are met and thus we have a local minimizer. Since $f$ and $g$ are convex, our local minimizer (solution) is also an optimal solution.
			\item Since we have an optimal solution, the KKT conditions have been met, thus $\boldsymbol{\mu}_0 g(\boldsymbol{x}_0) = 0$. This time we need to show $f(\boldsymbol{x}_0) = f(\boldsymbol{x}_0) + \boldsymbol{\mu}_0 g(\boldsymbol{x})$. The last term, as we just said, is 0, so we have $f(\boldsymbol{x}_0) = f(\boldsymbol{x}_0)$.
		\end{enumerate}
\end{enumerate}

\end{document}