\documentclass{article}
\usepackage[top=1in, bottom=1in, left=1in, right=1in]{geometry}
\usepackage{amsmath}
\usepackage{graphicx}
\begin{document}

\begin{flushright}
Matt Jibson \\
ST 521 \\
HW 4
\end{flushright}

\begin{enumerate}
	\item %1
		$P_{00}^n > 0$ for all $n$. State 1 is not periodic as it can only be the current state during the initial state. $P_{22}^n > 0$ for $n \ge 4$. $P_{33}^n > 0$ for $n \ge 3$. $P_{44}^n > 0$ for $n \ge 2$. $P_{55}^n > 0$ for $n \ge 1$. Hence $d(i) = 1$ for all states.
	\item %2
		States 1, 2, 3, and 4 are recurrent since once we are in states $\{1, 3\}$ or $\{2, 4\}$, we cannot leave the group. States 5 and 6 are transient since there is a $> 0$ chance that we will enter the other 4 states, which would mean we will never return to state 5 or 6.
	\item %3
		For $r = 1, P_{ii} = 1$, states $\ge 1$ are recurrent, positive, and aperiodic. Since $f_{ii}(0) = 1, f_{ii}(n) = 0$ for $n > 0$, $\mu_i = 1$. State 0 is transient (assuming $a_0 \ne 1$) hence $\mu_0 = \infty$. If $a_0 = 1$, state 0 acts like the other states.

		For $r < 1$ states $\ge 1$ are recurrent and aperiodic.
	\item %4
		At $p = 0$, $P$ is the identity matrix and all states are recurrent, positive (since $f_{ii}(0) = 1, f_{ii}(n) = 0, n > 0$, hence $\mu_i = 1$), and aperiodic (since $d(i) = 1$), so the states are ergodic.

		At $p = 0.25$ (analysis is the same for $0 < p < 0.5$),
		\begin{displaymath}
			P = \left( \begin{array}{ccc} 0.5 & 0.5 & 0 \\ 0.25 & 0.5 & 0.25 \\ 0 & 0.5 & 0.5 \end{array} \right)
		\end{displaymath}
		and all states are recurrent. $P_{ii}^n > 0$ for all $n$, so states are aperiodic.

		%Diagonalizing $P$ gives us
		%\begin{displaymath}
		%	D = \left( \begin{array}{ccc} 0 & 0 & 0 \\ 0 & 1 & 0 \\ 0 & 0 & 0.5 \end{array} \right)
		%\end{displaymath}
	\item %5
		$f_{00}(1) = 1 - a, f_{00}(2) = ab, f_{00}(3) = (1-a) ab, f_{00}(4) = (1-a)^2 ab$. Assuming $0 < a, b < 1$, both states are clearly recurrent. Hence for $n \ge 2, f_{00}(n) = (1-a)^{n-2} ab$. Since $P_{00}^n \not \rightarrow 0$ as $n \rightarrow \infty$, state 0 is positive and hence $\mu_i < \infty$. $\mu_0 = E(T_0|X_0=0) = \Sigma_{n=1}^\infty f_{00}(n) = (1-a) + \Sigma_{n=2}^\infty (1-a)^{n-2} ab$.
	\item %6
		If all other states $i$ not $s$ communicate with $s$, then for some $n \ge 1, p_{is}^n > 0$, hence all states have a $> 0$ probability of jumping to $s$, and hence being absorbed. When this happens, $P(X_n = i | X_0 = i) = 0$, hence all states other than $s$ are transient.
	\item %7
		If and only if a state $i$ is transient then for some $n \ge 1, P(X_n = i | X_0 = i) < 1$, hence there were a finite number of visits to $i$ and so the mean number of visits to $i$ will be finite. Thus state $i$ is recurrent if and only if for some $n \ge 1, P(X_n = i | X_0 = i) = 1$, hence there will always be another visit to $i$, so the mean number of visits is infinite.
\end{enumerate}

\end{document}