Fico muito feliz quando vejo mission\'arios tendo a oportunidade de servir em posi\c c\~ao de lideran\c ca em nossa miss\~ao. Sua dignidade e desejo de servir foi visto pelo Senhor. Senti a inspira\c c\~ao do Senhor para cham\'a-lo para servir na posi\c c\~ao de \textbf{L\'ider de Distrito} em nossa miss\~ao. Gostaria de lembr\'a-lo que \'e o Senhor que o chama. Seu encargo como l\'ider de distrito \'e de ser um exemplo de lealdade ao Senhor. Seus manuais de instru\c c\~oes ser\~ao ``Pregar Meu Evangelho'' e o manual mission\'ario. Preste aten\c c\~ao especialmente \`as partes sobre a reuni\~ao de distrito em ``Pregar Meu Evangelho'', e sobre Lideran\c ca no Manual Mission\'ario. Voc\^e tem a responsabilidade de ensinar os membros do seu distrito os princ\'ipios que se encontram nestes manuais. A melhor maneira que voc\^e pode fazer isso \'e dividir com os membros do seu distrito, e mostrar atrav\'es do seu exemplo e preceito como fazer a obra mission\'aria com maior efic\'acia. E um exemplo de efic\'acia \'e cumprir com as metas de batismos de seu distrito.

Voc\^e tamb\'em tem a responsabilidade de planejar e preparar reuni\~oes do distrito que v\~ao ajudar os mission\'arios a melhorarem a cada semana. Estas reuni\~oes devem durar 60--90 minutos e pelo menos a metade do tempo ser\'a para pr\'atica das t\'ecnicas do trabalho mission\'ario. Voc\^e tamb\'em \'e respons\'avel para realizar entrevistas batismais em seu distrito e com os pesquisadores dos L\'ideres de Zona. Pe\c ca aos L\'ideres de Zona que forne\c ca treinamento sobre como realizar entrevistas batismais.

Elder Scott nos ensinou que ``n\~ao somos ju\'izes e sim treinadores''. N\~ao podemos ficar buscando os erros dos liderados para relatar para os L\'ideres de Zona ou Presidente. O L\'ider de Distrito deve resolver os desafios de seu distrito e se precisar de ajuda deve recorrer aos L\'ideres de Zona. Ame a todos os mission\'arios em seu Distrito, busque compreend\^e-los. Seja r\'apido em elogiar aquilo que \'e louv\'avel, e devagar em julgar ou criticar aqueles cujo desempenho n\~ao \'e o esperado. Ouvir e entender faz parte de uma lideran\c ca madura. Seja confi\'avel em todas as coisas. Confiamos em sua lealdade aos l\'ideres e programas da Igreja e no cumprimento fiel e exato de todas as instru\c c\~oes recebidas de seu Presidente de Miss\~ao. \textbf{\textit{``Sim, um homem cujo cora\c c\~ao transbordava de gratid\~ao a seu Deus pelos muitos privil\'egios e b\^en\c c\~aos que concedia a seu povo; um homem que trabalhava infatigavelmente pelo bem-estar e seguran\c ca do povo''}} (Alma 48:12). \'E isso o que o Senhor espera de um l\'ider.

Que o Senhor o aben\c coe em sua nova designa\c c\~ao na miss\~ao.
