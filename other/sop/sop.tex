\documentclass[12pt]{article}
\usepackage[top=1in, bottom=1in, left=1in, right=1in]{geometry}
\begin{document}

\begin{center}
Statement of Purpose \\
Matthew W. Jibson
\end{center}

I wish to be admitted to the graduate program in Computer-Based Music Theory and Acoustics at Stanford and to study at the Center for Computer Research in Music and Acoustics (CCRMA). I have gone through eight years of preparatory study in music and engineering and received degrees in both. The research I have done is exactly inline with the CCRMA and can be continued in some of the many opportunities represented there.

It seems that one of the purposes of the CCRMA is to add to the arsenal of instruments and techniques for musicians, composers, and engineers so that music of higher quality can be created. However, in order to identify the need and opportunity for this quality, a researcher must understand the qualities of music that make it beautiful. As far as I am aware, the only way to achieve this level of understanding is to have spent many hours with an instrument, be it piano, voice, or staff paper. The time spent in serious practice shows a musician how to sing and communicate with their audience. I have spent this time, and learned how to make my instruments sing. I know how to communicate emotional qualities of the music I am playing to my audience.

At the CCRMA, the other side of the coin is equally important: those working to advance the field need the tools and abilities to think in abstract solutions to difficult problems. As an electrical engineer I have learned some of these methods and the underlying mathematics. As a programmer I have been able to apply these to make real products that really work. For example, I have had internships with Seagate, IBM, and three other smaller companies. At each I created useful products using techniques learned from earlier schooling and self-directed education.

While pursuing a piano degree, I began to study organ with another professor. I found the engineering complexity of the instrument instantly appealing. An instrument that filled one side of a large room with over 2,000 pipes was controlled completely without electronics. Complex interactions where manuals couple to other manuals were possible. The great number of registration combinations possible with which to voice the music intrigued me. What were the governing principles behind combining reeds and flutes? Why were the pipes positioned on alternating sides, and the pipes of each manual contained in one place? What were the physics of a pipe that produce sound? The principles governing the organ seemed complex, but quantifiable. Yet, the principle that always returned during a lesson was the same. Where are the phrases and articulations that allowed the melody to sing to the listener? How did I create music out of the black marks on the page?

To answer these questions, which would allow me to better understand organs, I worked with two other students also double-majoring in engineering and music to build a hybrid pipe and electronic organ as our senior project. We had an electrical, computer (me), and mechanical engineer, each with appropriate tasks for our project. I was in charge of the programming and logic to receive input from the keyboard and send output to the pipe valves and an audio signal to the speakers. During our work, I had a desire to create a high quality, synthetic pipe organ sound. Using analysis methods and mathematics from earlier courses, I came up with a solution. By using the discrete Fourier transform I was able to analyze the harmonic qualities of some recordings we made of the pipe organ on campus. Interpolation yielded functions to describe the harmonic structure of waves at arbitrary frequencies. (Refer to the accompanying writing sample for more detail.) Applying the mathematics of signals to analyze the data, I wrote code to implement the analysis and produce usable output, and integrate that output into a field-programmable gate array. Yet, I did not lose sight of why I was doing this: to increase the available tools to advance the art and raise the enjoyment of great music. For example, the same day that we had to present our senior projects I also had my senior piano recital. Part of my warm-up was to sit for a few hours in the presentation room and play duets on our organ with the mechanical engineer of the group (also a piano major). As is often the case, this act of reading through some simple, beautiful music repeated again to my mind why good music is good.

The fundamental problem with the analysis work I did is that there was no model to describe the change in the order of the harmonics. Since organs are made of many pipes, each pipe produces its own unique harmonic structure. Simple interpolations cannot be used to describe changes across the entire range. (This concept is not easy or quick to describe, and so has been done at greater length in the accompanying writing sample.) This is the problem I wish to address if admitted to the CCRMA. I want to find underlying models that can describe harmonic changes. The math for this, I believe, has not yet been researched. Most of the work on organ synthesis so far has been on acoustics, room, and pipe modeling. I have not found work that synthesizes the organ as a whole, and takes into account its uniqueness at having individually sculpted pipes for each sound it can make. The solution space for this problem is large. For example, linear programming and optimization techniques could present a good solution, since problems with a large search space work well there. Perhaps each harmonic is a variable or dimension, and so the formation of this solution would involve some transformation into a simplex-appropriate or convex problem. Perhaps physical modeling can be used where each frequency inputs different parameters into the model, producing the desired unique harmonic structure for any frequency. Solutions may also lie in time-domain analysis, acoustical physics, or some other unconsidered path. The CCRMA, with its broad range of faculty and interests, will provide me the opportunity and ability to make this a well-defined problem that is in the solvable domain. Then again, this research could take me in unexpected ways that are not now anticipated.

I have been preparing for an experience like the CCRMA for many years. My passions lie equally in both fields of music and engineering. I do not desire to be a musician who can program, or an engineer who can play organ. Professionally I will be able to contribute to the field of electronic music, though where or how is, of course, too far away to tell. The CCRMA will allow me to spend the time needed to research this difficult problem and create solutions that will contribute to the knowledge of music analysis. Most important, this work will enable musicians to produce better, more expressive music.

\end{document}
