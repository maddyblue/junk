\documentclass[10pt, xcolor=pdftex,dvipsnames,table]{beamer}
\usepackage[utf8]{inputenc}
\usepackage{color}

\mode<beamer>{
	\usetheme{Frankfurt}
	\usenavigationsymbolstemplate{}
}

\usepackage{calc}
\usepackage{ifthen}
\usepackage{tikz}


\title{Raio-X}
\subtitle{ {{ month }} {{ year }} }
\institute{Missão Brasil Rio de Janeiro}
\date{}


\begin{document}

\definecolor{criancas}{rgb}{1, 0.64, 0}
\definecolor{mocas}{rgb}{1, 0.2, 0.6}
\definecolor{rapazes}{rgb}{0.2, 0.6, 1}
\definecolor{mulheres}{rgb}{0.8, 1, 0}
\definecolor{homens}{rgb}{1, 0, 0}

\newcommand{\slice}[5]{
  \pgfmathparse{0.5*#1+0.5*#2}
  \let\midangle\pgfmathresult

  % slice
  \draw[thick,fill=#5!90] (0,0) -- (#1:1) arc (#1:#2:1) -- cycle;

  % outer label
  \node[label=\midangle:#4] at (\midangle:1) {};

  % inner label
  \pgfmathparse{min((#2-#1-10)/110*(-0.3),0)}
  \let\temp\pgfmathresult
  \pgfmathparse{max(\temp,-0.5) + 0.8}
  \let\innerpos\pgfmathresult
  \node at (\midangle:\innerpos) {#3};
}


\begin{frame}
	\titlepage
\end{frame}


\section{Raio-X : Missão Brasil Rio de Janeiro : {{ month }} {{ year }}}

\newcounter{a}
\newcounter{b}


	


\end{document}
