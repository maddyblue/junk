\documentclass{article}
\usepackage[top=1in, bottom=1in, left=1in, right=1in]{geometry}
\usepackage{graphicx}
\usepackage{amsmath}
\begin{document}

\begin{flushright}
Matt Jibson \\
ECE 501 \\
HW 1
\end{flushright}

\begin{itemize}
	\item[1.2]
		As defined, a complex system is one that involves some amount of multiple disciplines, and advanced (or sometimes just not well understood) technology, regardless of their engineering status. Thus, the primary difference between an engineered and non-engineered complex system is that the latter is not designed by man. It very well may be influenced or created by man, such as eco-systems and social structures, and we may try to study it, such as living organisms, but we are not responsible for its operation. Our national government is a good example of where we have tried to apply systems engineering principles to a non-engineered complex system. We have cabinet members, advisors, and an executive. Here each participant tries to create models of input and output for their branch (economy, military, etc.), and show how they interact with the others, so we can get some degree of predictability in the system of our country, which must be applied by an overseeing manager who can balance all advice, knowing that each is an expert in only their field.

	\item[1.5] \
		\begin{itemize}
		\item[Pro 1.] New technology enables new features, for example, computer-based stability controls in a car that take many inputs (individual wheel speed, braking, acceleration, weather, movement direction) and adjust wheel power to increase stability in turns, on ice, etc.
		\item[Con 1.] New technology, by definition, has not been as thoroughly tested, and thus may be dangerous and not well understood. Take the previous example, which, if not done correctly, can lead to the car making bad decisions leading to a crash.
		\item[Pro 2.] New technology allows legacy systems to be replaced, removing known problems and issues. For example, an Intel chip some years ago had a known bug in its division calculation that programmers had to work around. With newer chips, the bug was fixed and could be ignored.
		\item[Con 2.] New technology can be expensive, leading to monetary risks that may not always return on investment. For example, there are many technologies available to chip manufactures that increase performance, but at a great cost that is believed to not be cost effective. Yet, some of these risks must be taken lest no progress is made.
		\end{itemize}

	\item[1.6]
		Modularity means that components of the system are designed to have a simple interface to other components to provide easy interaction, test, maintenance, and reliability. A modular system posses characteristics of a layered design. Each layer interacts with another in a known way, and does not over step its bounds in that operation. Thus, work and replacement may easily be done on each layer without changes to the surrounding systems, as long as that new layer or module conforms to the specification. One example of a modular system is a computer. A computer has a defined operation for each of its parts: motherboard, CPU, RAM, peripherals. As long as a device operates according to the spec, it may replace a component or be added to the system.

	\item[1.7]
		One pair is complexity and time. The more complex a system is, the longer it takes to ensure correct operation. Another pair is newness and availability. When very new technologies are desired, it can be difficult to find resources (as in physical for parts and talent for the engineering) to correctly implement the new systems.
\end{itemize}

\end{document}